% Source: https://www.overleaf.com/latex/templates/math-notes-and-homeworks-template/pqpxbnjjdtcs

\documentclass[12pt]{article}
 
\usepackage[margin=1in]{geometry} 
\usepackage{amsmath,amsthm,amssymb,graphicx,mathtools,tikz,hyperref}
\usetikzlibrary{positioning}
\newcommand{\n}{\mathbb{N}}
\newcommand{\z}{\mathbb{Z}}
\newcommand{\q}{\mathbb{Q}}
\newcommand{\cx}{\mathbb{C}}
\newcommand{\real}{\mathbb{R}}
\newcommand{\field}{\mathbb{F}}
\newcommand{\ita}[1]{\textit{#1}}
\newcommand{\com}[2]{#1\backslash#2}
\newcommand{\oneton}{\{1,2,3,...,n\}}
\newcommand\idea[1]{\begin{gather*}#1\end{gather*}}
\newcommand\ef{\ita{f} }
\newcommand\eff{\ita{f}}
\newcommand\proofs[1]{\begin{proof}#1\end{proof}}
\newcommand\inv[1]{#1^{-1}}
\newcommand\setb[1]{\{#1\}}
\newcommand\en{\ita{n }}
\newcommand{\vbrack}[1]{\langle #1\rangle}

\newenvironment{definition}[2][Definition]{\begin{trivlist}
\item[\hskip \labelsep {\bfseries #1}\hskip \labelsep {\bfseries #2.}]}{\end{trivlist}}
\newenvironment{theorem}[2][Theorem]{\begin{trivlist}
\item[\hskip \labelsep {\bfseries #1}\hskip \labelsep {\bfseries #2.}]}{\end{trivlist}}
\newenvironment{lemma}[2][Lemma]{\begin{trivlist}
\item[\hskip \labelsep {\bfseries #1}\hskip \labelsep {\bfseries #2.}]}{\end{trivlist}}
\newenvironment{exercise}[2][Exercise]{\begin{trivlist}
\item[\hskip \labelsep {\bfseries #1}\hskip \labelsep {\bfseries #2.}]}{\end{trivlist}}
\newenvironment{remark}[2][Remark]{\begin{trivlist}
\item[\hskip \labelsep {\bfseries #1}\hskip \labelsep {\bfseries #2.}]}{\end{trivlist}}
\newenvironment{proposition}[2][Proposition]{\begin{trivlist}
\item[\hskip \labelsep {\bfseries #1}\hskip \labelsep {\bfseries #2.}]}{\end{trivlist}}
\newenvironment{corollary}[2][Corollary]{\begin{trivlist}
\item[\hskip \labelsep {\bfseries #1}\hskip \labelsep {\bfseries #2.}]}{\end{trivlist}}
 \hypersetup{
 colorlinks,
 linkcolor=blue
 }
\begin{document}
\date{}

 
\title{Topology II}
\author{Jose Iovino\\Professor \\UTSA} 
 
\maketitle
\section{Filters}
\subsection{Basic Notions}
Topology is the study of limits and convergence.  Filters are important because they generalize the notion of limit.  Intuitively, a filter is collection of ``large" sets.  Let $I$ be a set.  We claim a subset of $I$ is ``large" if it contains ``almost all" of the elements of $I$. 

Conditions that ``large" subsets must satisfy.
\begin{itemize}
  \item $I$ is a large subset of itself, since it contains ``all" of the elements.
  \item An intersection of large subsets should be large.
  \item If $A$ is a large subset of $I$, then any $B \supseteq A$ is large.
\end{itemize}

\begin{definition}{filter}
Let $I$ be any set.  A filter $\mathcal{F}$ on $I$ is a collection of subsets of $I$ such that
\begin{enumerate}
    \item $I \in \mathcal{F}$ and $\emptyset \not\in \mathcal{F}$.
    \item If $A, B \in \mathcal{F}$, then $A \cap B \in \mathcal{F}$.
    \item If $A \in \mathcal{F}$ and $B \supseteq A$, then $B \in \mathcal{F}$.
\end{enumerate}
\end{definition}

The universal quantifier ``$\forall$" says a property holds for all elements, while the existential quantifier ``$\exists$" says a property holds for at least element.  Filters enable the definition of quantifiers between these two extremes which capture the notion of a property holding for ``almost all" elements.

\begin{definition}{almost all}
If $\mathcal{F}$ is a filter on $I$ and $P$ is a property.  We say that almost all elements of $I$ satisfy $P$ when
\[
\{ x \in I | x \textrm{ satisfies } P\} \in \mathcal{F}.
\]
\end{definition}

\subsection{Examples}
\subsubsection{Co-finite Filter on $\mathbb{N}$}
On $\mathbb{N}$, let
\[\mathcal{F} = \{A \subset \mathbb{N} | A^c \textrm{ is finite} \}.
\]
This is the co-finite filter.  Check that it is indeed a filter.

\begin{enumerate}
    \item $\mathbb{N} \in \mathcal{F}$ because $\mathbb{N}^c = \emptyset$.
    \item $A, B \in \mathbb{F}$, then $A^c, B^c$ are finite.  By De Morgan, $A^c \cup B^c = (A \cap B)^c$ is finite.  Hence $A \cap B \in \mathcal{F}$.
    \item If $A \in \mathcal{F}$ and $B \superseteq A$, then $B^c \subseteq A^c$ which is finite.  Hence, $B \in \mathcal{F}$.
\end{enumerate}

\subsubsection{Frechet Filter}
In the previous example, none of the properties used where special to the natural numbers.  Thus we can generalize to any infinite index set $I$.  Then
\[
\mathcal{F} = \{ A \subseteq I | A^c \textrm{ is finite} \}
\]
is a filter.  It is called the Frechet filter on I.

\subsubsection{Co-countable Filter}
This example can be extended even more by playing with the cardinalities involved.  For example, if $I$ is an uncountable set and
\[
\mathcal{F} = \{ A \subseteq I| A^c \textrm{ is countable}\}
\]
then $\mathcal{F}$ is the co-countable filter.

\subsubsection{A Filter from Measure Theory}
Let $(X, \Sigma, \mu)$ be a probability space, then 
\[
\mathcal{F} = \{ A \subseteq X | \mu(A) = 1 \}
\]
is a filter.  It consists of the events which are certain to happen.

\subsubsection{Trivial Example}
If $X$ is a set, then
\[
\mathcal{F} = \{ X \}
\]
is a filter on $X$.  This is called the trivial filter on $X$.

\section{Limits}
Let $X$ be a topological space.  Let $(x_i)_{i \in I}$ be a family of elements of $X$ and let $a \in X$.  If $\mathcal{F}$ is a filter on $I$, we write
\[
x_i \xrightarrow{\mathcal{F}} y
\]
if $\forall$ neighborhoods $U$ of $a$
\[
\{ i | x_i \in U \} \in \mathcal{F}.
\]

\begin{remark}{}
If $I = \mathbb{N}$ and $\mathcal{F}$ is the Frechet filter, then for any sequence $(x_n)_{n \in \mathbb{N}}$,
\[
x_n \xrightarrow{\mathcal{F}} a
\]
if and only if $x_n \rightarrow a$ as $n \rightarrow \infty$ in the classical sense.
\end{remark}

\begin{definition}{Ultrafilter}
If $I$ is an index set then an ultrafilter $\mathcal{F}$ on $I$ is a filter such that $\forall A \subseteq I$, either $A \in \mathcal{F}$ or $A^c \in \mathcal{F}$.
\end{definition}

\begin{theorem}
If $I$ is any set and $\mathcal{F}$ is any filter on $I$, then there exists an ultrafilter $\mathcal{U}$ on $I$ such that $\mathcal{F} \subseteq \mathcal{U}$.
\end{theorem}
\end{document}